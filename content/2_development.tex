\chapter{Development in Ion Shell}

\section{Finding areas for contribution}

With the goal to contribute to Ion shell during the period of the it project, the possible bug fixes/features needed to be determined first.
The following activities were undertaken to find out concrete tasks to accomplish:

- Going through the online manual of Ion Shell.
- Reading posted issues on the GitLab Repository of Ion shell.

The online manual of the shell is meant for the users of the application.
It focuses on the following points:

- What and how something can be done in an interactive session.
- Features of the scripting language.
- Philosophy the application. Why does it exits and what it is supposed to be.

Investigating the online manual of Ion shell provided the overview of already implemented features and how the shell was meant to be used.
While reading the manual, it became clear that the documentation lacked explanations in certain areas.
Later on additions to the documentation were also required for new implemented features in Ion Shell.
Because of this the manual presents one of the fields in Ion shell for contribution.

Another important source for the deduction of tasks is the collection of issues and pending pull requests from a repository.
This investigation yielded a number of goals to undertake.
However it also pointed out that the development of this project stagnated for a while.
Over the course of the it project I would be the only contributor, most of time.

\section{Automated testing of features}

Unit tests and integration tests are the 2 forms of tests used to verify the functionality of the shell.
The observation of the code base suggests that the first form as unit tests are mainly used to inspect the correctness of tokenisation, parsing and
evaluation of the builtin functions of the shell in isolation.
It is important to note that these unit tests execute and assert inner/private functions too
in contrast to other programming languages like CSharp where unit test should not access private functionality easily.
This is typical in the Rust language because unit tests are often in the children module in respect of the code to be tested.
Visibility in Rust enables code in a children module to have full access to all functionality in a parent module.
This allows to test private functions in Rust easily \cite{rust_testing_private_function}.

Integration tests vary depending on the kind of application.
Ion shell's approach to integration test is the execution of script files
and comparing the output to text files which contain the expected output for the test.
These testing does therefore focus on the scripting aspect of the shell.
Especially this kind of testing is valued highly according to the Contributing Guidelines of this porject \cite{ion_shell_contributing}.
Since rust does not provide a builtin way to perform these kinds of tests,
a rather complex bash script was written to orchestrate this kind of integration tests.
Before this it project there was no explanation of how these new integration tests are executed and how they should be written.
It was however paramount to add new integration tests to verify bug fixes and implemented features.
A documentation section in the contributing guideline was crafted during the process of
learning the working of this integration tests \cite{ion_shell_contributing}.
This will hopefully reduce the learning effort for future contributors in the area of integration tests.

\section{Example of development}

\subsection{Implementing an array method in Ion Shell}

The feature Parameter Substitution via the array method "subst" ought to represent in the development work in Ion Shell.
The following tasks were performed to archive the implementation:

- Determine the desired behaviour of the array method.
- Create a test to verify the functionality of the implementation.
- Implement the method in Ion shell.
- Document this feature in the online manual of Ion shell.

The implementation was proposed and merged under the Pull Request \cite{pr_subst_method}.

\subsubsection{Finding out the desired behaviour of "subst"}

This feature was requested in an issue a certain while ago \cite{ion_shell_subst_issue}.
In this issue the specifics of behaviour of this array method have already been discussed too.
The discussion concluded the following signature of "subst".

```text
subst(input: T, default: T) -> T
```
source: \cite{issue_subst_request}

Where the invariant says that T must be of type array.
This array method takes two arguments as arrays and returns either the first or second argument.
The first one is returned if it is not an empty array. Otherwise the second argument is returned.
The accomplished implementation follows the signature and invariant.

\subsection{Creating integration test for the implementation}\label{subst_test}

To verify the functionality of the implementation, an integration test was added to test suit.
An integration test consists of 2 files. One is an ion file which is executed by the shell an other is a plain text file with extension "out".
This text file describes the expected output of the execution of the ion test file.
For this implementation there is the file "subst.ion" and "subst.out"
where "subst" is the name of the test. Both file must share the same base file name, here "subst".

The code snippet 1.1 represents the test code.

\begin{lstlisting}[language=sh,caption={integration test for subst method. Source: \cite{ion_shell_subst_ion_test}},captionpos=b]

let array = []

# Inline array in the method
echo @subst(@array [foo bar])

# single value
echo @subst(@array [baz])

# variable expansion
let default = [foobar]
echo @subst(@array @default)

# method would not trigger
let array ++= faz
echo @subst(@array @default)

for number in  @subst([] [2 3])
  echo $number
end

\end{lstlisting}

\subsection{Implementation of "subst"}

\subsection{Documentationa of "subst"}
